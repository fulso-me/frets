\def\frets{15} % Number of frets to draw, markers get put at 3, 5, 7, 9, 15, 17 and double markers on 12

% \Ukelele{}
% \BaritoneUkelele{}
\Guitar{}
% Inside guitar, you can define new instruments like this:
%   \def\strings{6}
%   \def\tuning{1/E, 2/B, 3/G, 4/D, 5/A, 6/E}
%   \def\rootNote{\getname{E}} % Highest string, same as 1 in \tuning


\drawLines{} % drawLines is mandatory
\drawDots{} % draw the fret markers
\drawNumbers{} % draw numbers over frets
% \drawTuning{} % draw the instrument tuning on the nut

% \def\n{C} % You can define variables to use
% Example: \drawMajorTriad{\n}

% CAGED: draw labels for CAGED for the given scale. You have on option between five layers to draw on Label1-Label5
% \drawCShapeLabel{C}{Label5}
% \drawAShapeLabel{C}{Label4}
% \drawGShapeLabel{C}{Label3}
% \drawEShapeLabel{C}{Label2}
% \drawDShapeLabel{C}{Label1}

% \drawNote{B} % Draw the note
% \drawNoteAs{B}{M}{znode} % Draw note with specifications. B is the note, M is what to write, znode is how to write it. (znode is black, hnode is grey, bnode is blue)

% \drawChord{G}{{1,5,8}} % Draw all notes in the chord, #1 is the root, {1,5,8} are the notes you would like. {1,5,8} is a major triad.
% \drawMajorTriad{C} % Draw all the notes from the major triad
% \drawMinorTriad{D} % Draw all the notes from the minor triad
% \drawSeventh{D} % Draw all the notes from the seventh chord
% \drawSixth{D} % Draw all the notes from the Sixth chord

% \drawPentatonic{G} % Draw all relative notes in the pentatonic as numbers (i, ii, iii, v, vi). i is noted as M (major), vi is noted as m (minor)
% \drawScale{G} % Draw all relative notes in the major scale as numbers (i, ii, iii, iv, v, vi, vii). i is notes as M (major), vi is noted as m (minor)
% \drawMajorBlues{G} % Draw all relative notes in the major blues scale as numbers (i, ii, b3, iii, v, vi). i is notes as M (major), vi is noted as m (minor), biii is noted as b3
