\begin{tikzpicture}[
    ynode/.style={draw=red!50,circle,fill=red!50,scale=.35,inner sep=1pt,minimum size=1.7em},
    zbar/.style={shorten >=-3,shorten <=-3,line width=6,round cap-round cap},
    znode/.style={white,draw=black,circle,fill=black,scale=.5,inner sep=1pt,minimum size=1.7em},
    hnode/.style={white,draw=black,circle,fill=gray,scale=.5,inner sep=1pt,minimum size=1.7em},
    bnode/.style={white,draw=blue!50,circle,fill=blue!50,scale=.5,inner sep=1pt,minimum size=1.7em},
    ]

  \newcommand\savename[2]{\expandafter\xdef\csname name#1\endcsname{#2}};
  \newcommand\getname[1]{\csname name#1\endcsname};
  \foreach \n/\t in {0/A,1/Bb,2/B,3/C,4/Db,5/D,6/Eb,7/E,8/F,9/Gb,10/G,11/Ab}{
    \savename{\n}{\t};
    \savename{\t}{\n};
  }

\newcommand{\defRoot}{
  \foreach \str/\tune [count=\ni] in \tuning {
    \ifthenelse{\ni = 1}{
      \def\rootNote{\getname{\tune}}}{}
  }
}

\newcommand{\Ukelele}{
  \def\strings{4}
  \def\tuning{1/A, 2/E, 3/C, 4/G}
  \def\rootNote{\getname{A}}
}
\newcommand{\BaritoneUkelele}{
  \def\strings{4}
  \def\tuning{1/E, 2/B, 3/G, 4/D}
  \def\rootNote{\getname{E}}
}
\newcommand{\Guitar}{
  \def\strings{6}
  \def\tuning{1/E, 2/B, 3/G, 4/D, 5/A, 6/E}
  \def\rootNote{\getname{E}}
}

\newcommand{\drawLines}{  
%%%% Draw the base and set coordinates %%%%
  \begin{scope}[xscale=-15,yscale=.3,line width=.5]

    \xdef\x{1}
    %% Left line
    \draw[line width=1.5] (1,1) -- (1,\strings);
    \foreach \fret in {1,...,\frets}{
      %% Set coordinate for each string
      \foreach \str in {1,...,\strings}{
        % Reverse strings so 1 is on top
        \coordinate (\str-\fret) at (0.97193715634*\x,\strings+1-\str);
      }
      %% Set coordinate for the text above
      \coordinate (Top-\fret) at (0.97193715634*\x,\strings+1);
      \coordinate (Label1-\fret) at (0.97193715634*\x,\strings+2);
      \coordinate (Label1-0) at (1,\strings+2);
      \coordinate (Label2-\fret) at (0.97193715634*\x,\strings+3);
      \coordinate (Label2-0) at (1,\strings+3);
      \coordinate (Label3-\fret) at (0.97193715634*\x,\strings+4);
      \coordinate (Label3-0) at (1,\strings+4);
      \coordinate (Label4-\fret) at (0.97193715634*\x,\strings+5);
      \coordinate (Label4-0) at (1,\strings+5);
      \coordinate (Label5-\fret) at (0.97193715634*\x,\strings+6);
      \coordinate (Label5-0) at (1,\strings+6);
      %% Compute the position of the fret
      \pgfmathsetmacro\x{\x * 0.94387431268}
      \xdef\x{\x}
      %% Draw the fret
      \draw (\x,1) -- (\x,\strings);
    }

    %% Draw each string
    \foreach \str in {1,...,\strings}{
      \draw (1,\str) -- (0.97153194115*\x,\str);
        % Reverse strings so 1 is on top
      \coordinate (start\str) at (1,\strings+1-\str);
      \coordinate (\str-0) at (1,\strings+1-\str);
    }
  \end{scope}
}

  \newcommand{\drawDots}{
    %% Draw the mark on the guitare
    \pgfmathtruncatemacro\lowMidString{\strings / 2}
    \pgfmathtruncatemacro\highMidString{(\strings / 2)+1}

    \ifthenelse{\frets > 16}{\def\dotlist{3,5,7,9,15,17}}{
    \ifthenelse{\frets > 14}{\def\dotlist{3,5,7,9,15}}{
    \ifthenelse{\frets > 8}{\def\dotlist{3,5,7,9}}{
    \ifthenelse{\frets > 6}{\def\dotlist{3,5,7}}{
    \ifthenelse{\frets > 4}{\def\dotlist{3,5}}{
    \ifthenelse{\frets > 2}{\def\dotlist{3}}{
      \def\dotlist{{}}
    } } } } } }

    \foreach \f in \dotlist{
      \draw[black!20,fill=black!10] ($(\lowMidString-\f)!.5!(\highMidString-\f)$) circle (.08);
    }
    \ifthenelse{\frets > 11}{
      \draw[opacity=.20,fill,fill opacity=.10] (\lowMidString-12) circle (.08) (\highMidString-12) circle (.08);
    }{}
  }

  \newcommand{\drawNumbers}{
    %% Number above each space
    \foreach \fret in {1,...,\frets}{
      \node[scale=.8] at (Top-\fret) {\tiny \fret};
    }
  }

  \newcommand{\drawTuning}{
    % draw tuning
    \foreach \str/\name in \tuning{
      \node[znode] at (start\str) {\textbf{\name}};
    }
  }

  %\node[znode] at (4-4) {\textbf{2}};
  %\node[znode] at (3-5) {\textbf{4}};
  %\draw[zbar] (2-3) -- (4-3);

  %\node[scale=.8] at (Top-1) {G}; \node[znode] at (start1) {\textbf{m}}; \node[znode] at (start2) {\textbf{iii}}; \node[znode] at (start3) {\textbf{M}}; \node[znode] at (start4) {\textbf{v}}; \node[znode] at (1-3) {\textbf{M}}; \node[znode] at (2-3) {\textbf{v}}; \node[znode] at (3-2) {\textbf{ii}}; \node[znode] at (4-2) {\textbf{m}};

  \newcommand{\drawNode}[4]{
    \pgfmathtruncatemacro\mirror{#3+12}
    \pgfmathtruncatemacro\fretlimit{\frets+1}
    \ifthenelse{#3 < \fretlimit}{
      \node[#1] at (#2-#3) {#4};
    }{}
    \ifthenelse{\mirror < \fretlimit}{
      \node[#1] at (#2-\mirror) {#4};
    }{}
  }

  \newcommand{\defineFrets}[1]{
    \def\firstFret{#1};
    \pgfmathtruncatemacro\secondFret{mod(\firstFret+1,12)};
    \pgfmathtruncatemacro\thirdFret{mod(\firstFret+2,12)};
    \pgfmathtruncatemacro\fourthFret{mod(\firstFret+3,12)};
    \pgfmathtruncatemacro\fifthFret{mod(\firstFret+4,12)};
  }

  \newcommand{\drawCShapeLabel}[2]{
    \pgfmathtruncatemacro\foundRoot{mod(\getname{#1}+(12-\rootNote),12)}
    \foreach \offset in {5,...,8}{
      \pgfmathtruncatemacro\offsetRoot{mod(\foundRoot-\offset,12)}
      \drawNode{scale=.8}{#2}{\offsetRoot}{C};
    }
  }
  \newcommand{\drawAShapeLabel}[2]{
    \pgfmathtruncatemacro\foundRoot{mod(\getname{#1}+(12-\rootNote),12)}
    \foreach \offset in {3,...,6}{
      \pgfmathtruncatemacro\offsetRoot{mod(\foundRoot-\offset,12)}
      \drawNode{scale=.8}{#2}{\offsetRoot}{A};
    }
  }
  \newcommand{\drawGShapeLabel}[2]{
    \pgfmathtruncatemacro\foundRoot{mod(\getname{#1}+(12-\rootNote),12)}
    \foreach \offset in {0,...,3}{
      \pgfmathtruncatemacro\offsetRoot{mod(\foundRoot-\offset,12)}
      \drawNode{scale=.8}{#2}{\offsetRoot}{G};
    }
  }

  \newcommand{\drawEShapeLabel}[2]{
    \pgfmathtruncatemacro\foundRoot{mod(\getname{#1}+(12-\rootNote),12)}
    \foreach \offset in {-1,...,2}{
      \pgfmathtruncatemacro\offsetRoot{mod(\foundRoot+\offset,12)}
      \drawNode{scale=.8}{#2}{\offsetRoot}{E};
    }
  }

  \newcommand{\drawDShapeLabel}[2]{
    \pgfmathtruncatemacro\foundRoot{mod(\getname{#1}+(12-\rootNote),12)}
    \foreach \offset in {1,...,5}{
      \pgfmathtruncatemacro\offsetRoot{mod(\foundRoot+\offset,12)}
      \drawNode{scale=.8}{#2}{\offsetRoot}{D};
    }
  }

  % \newcommand{\drawShapeName}[2]{
  %   \defineFrets{#1};
  %   \drawNode{scale=.8}{Label}{\thirdFret}{#2};
  %   }

  % \newcommand{\drawShapeLabel}[1]{
  %   \pgfmathtruncatemacro\myRoot{mod(\getname{#1}-1,12)}
  %   \drawShapeName{\myRoot}{E}
  %   \pgfmathtruncatemacro\myRoot{mod(\myRoot+2,12)}
  %   \drawShapeName{\myRoot}{D}
  %   \pgfmathtruncatemacro\myRoot{mod(\myRoot+3,12)}
  %   \drawShapeName{\myRoot}{C}
  %   \pgfmathtruncatemacro\myRoot{mod(\myRoot+2,12)}
  %   \drawShapeName{\myRoot}{A}
  %   \pgfmathtruncatemacro\myRoot{mod(\myRoot+2,12)}
  %   \drawShapeName{\myRoot}{G}
  % }

  \newcommand{\drawChord}[2]{
    \foreach \chordNumber in #2{
      \pgfmathtruncatemacro\tuneNumber{mod(\getname{#1}+(\chordNumber-1),12)};
      \drawNote{\getname{\tuneNumber}}
    }
  }

  \newcommand{\drawNote}[1]{
    \pgfmathtruncatemacro\tuneNumber{\getname{#1}};
    \foreach \str/\tune in \tuning{
      \pgfmathtruncatemacro\curFret{mod(\tuneNumber+(12-\getname{\tune}), 12)}
      \drawNode{znode}{\str}{\curFret}{\textbf{#1}};
    }
  }
  \newcommand{\drawNoteAs}[3]{
    \pgfmathtruncatemacro\tuneNumber{\getname{#1}};
    \foreach \str/\tune in \tuning{
      \pgfmathtruncatemacro\curFret{mod(\tuneNumber+(12-\getname{\tune}), 12)}
      \drawNode{#3}{\str}{\curFret}{\textbf{#2}};
    }
  }

  \newcommand{\drawMajorTriad}[1]{
    \drawChord{#1}{{1, 5, 8}};
  }
  \newcommand{\drawMinorTriad}[1]{
    \drawChord{#1}{{1, 4, 8}};
  }
  \newcommand{\drawSeventh}[1]{
    \drawChord{#1}{{1, 5, 8, 11}};
  }
  \newcommand{\drawSixth}[1]{
    \drawChord{#1}{{1, 5, 8, 10}};
  }

  \newcommand{\drawPentatonic}[1]{
    \drawNoteAs{#1}{M}{znode}
    \pgfmathtruncatemacro\newNote{mod(\getname{#1}+2,12)}
    \drawNoteAs{\getname{\newNote}}{ii}{znode}
    \pgfmathtruncatemacro\newNote{mod(\getname{#1}+4,12)}
    \drawNoteAs{\getname{\newNote}}{iii}{znode}
    \pgfmathtruncatemacro\newNote{mod(\getname{#1}+7,12)}
    \drawNoteAs{\getname{\newNote}}{v}{znode}
    \pgfmathtruncatemacro\newNote{mod(\getname{#1}+9,12)}
    \drawNoteAs{\getname{\newNote}}{m}{znode}
  }

  \newcommand{\drawScale}[1]{
    \drawPentatonic{#1}
    \pgfmathtruncatemacro\newNote{mod(\getname{#1}+6,12)}
    \drawNoteAs{\getname{\newNote}}{iv}{hnode}
    \pgfmathtruncatemacro\newNote{mod(\getname{#1}+11,12)}
    \drawNoteAs{\getname{\newNote}}{vii}{hnode}
  }

  \newcommand{\drawMajorBlues}[1]{
    \drawPentatonic{#1}
    \pgfmathtruncatemacro\newNote{mod(\getname{#1}+3,12)}
    \drawNoteAs{\getname{\newNote}}{b3}{bnode}
  }
